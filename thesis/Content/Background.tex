%!TEX root = ../bachelors_thesis.tex
\chapter{Background}
The theory of Justification Logic as it is used here requires little knowledge of the wide fields of Modal Logic. For the purpose of this proof search a few basic rules and definitions are sufficient to provide the reader with the needed knowledge. 
\par
From a more practical angle a basic data structure was also needed to represent formulas. Already widely known and used \emph{Binary Syntax Trees} revealed to be just the perfect way to handle formulas.

\section{Justification Logic}
 The theory presented here is oriented mainly on the work of \cite{goet} as well as the older reference \cite{art} and also from the homepage \cite{stan}. This definitions and rules given here are not complete to the justification logic. Priority was given to those informations which are vital for the implementation. So however briefly and incomplete the theory is presented here full reference can be found in the named sources. 

\subsection{Origins}
Justification Logic has its origins from the field of modal logic. 
In model logic $\square A$ means that $A$ is \emph{know} or that we have \emph{proof} of $A$. In justification logic the equivalent would be $t:A$ where $t$ is a proof term of $A$. So we have the notion that \emph{knowledge} or \emph{proofs} may come from different sources. Justification logic lets us connect different \emph{proofs} with a few simple operators and thus describe better the proof. It may be said that where in model logic the knowledge is implicit it is explicit in Justification Logic\footnote{\cite{goet}}.

\subsection{Rules and Definitions}

\begin{definition}\footnote{\cite{goet} Page 17, incomplete}

\textbf{Justification terms} or just \textbf{terms} are syntactic objects given by the grammar
\[
	t::= c_{i}^{j} | x_i | (t \cdot t) | (t+t) | !t,
\]
where $i$ and $j$ range over positive natural numbers, $c_{i}^{j}$ denotes a (justification) constant of level $j$, and $x_i$ denotes a (justification) variable.

\end{definition}

\todoWriteMore{Todo: Finish this section!}

\section{Operation Syntax Tree}

\todoWriteMore{Todo}
