%!TEX root = ../../bachelors_thesis.tex
\section{Core Idea}
To search a formula for is provability it had to be found a way which allows to do the same steps, now matter what form the formula actually has. A first attempt was to strictly use recursion. This method should have worked but it proved to be very difficult to implement, because there are so many different cases to consider in one recursion step. Also the stack created by this could become problematic for very large formulas.

Instead a \emph{Divide and Conquer} approach is used. Diving will break even a large and complicated formula down to its most simple elements. Then these elements can be tested for their provability and in the conquer-step the results of the elements are put together giving the final result. Since the \emph{Divide and Conquer} algorithm design pattern uses multi-branched recursion there still remains some recursion but as this takes place at a much deeper level the cases within a recursion are reduced as well as the size of the recursion stack.

