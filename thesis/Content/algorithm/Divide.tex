%!TEX root = ../../bachelors_thesis.tex
\section{Divide}
The motivation behind the divide-step existed already long before the actual idea of the Divide and Conquer approach. Given a formula there would be no way to know what kind of formula it was or more precise: what operations were to be found within the \emph{justification term}. The original goal was to find a way to restructure any given formula so that handling it would always need the same steps and not depend to much on what the formula looks exactly. I was looking for something like the $CNF$\footnote{\emph{conjunctive normal form}, a conjunction of clauses, where a clause is a disjunction of literals} and use it in a similar way as $CNF$ is used in $PSC$\footnote{\emph{Proof Search Calculus} as it is introduced in \cite{jaeg}}. As the Sum-Rule for \emph{justification terms} works straight forward like a disjunction it is rather simple to restructure the formula as far as the Sum-Operator goes.\todoWriteMore{add graph or formula} But since the Multiplication-Operator is not even symmetric operator it does not work like the conjunction know from the $CNF$ and thus makes to restructuring of a formula all the more difficult. In addition there is the unary Bang-Operator which in itself is rather simple but still adds to the overall complexity.

In the end the restructuring would look like the following:
\begin{itemize}
	\item For each Sum-Operator in the formula, split it in two formulas.
	\item If the first operation of a formula is a Bang-Operation, check if it can be simplified. If not, remove this formula.
	\item There are certain positions of a Bang-Operator within the formula that cause the whole formula to be false. Those formulas shall be eliminated as well.
\end{itemize}
Those three steps which are called \emph{Atomize} in the source code break a given formula down to several simpler formulas which only contain the Multiplication-Operator as well as valid Bang-Operators. It is only then that a recursive method is called to analize the formula in a way that makes it possible to check if this subformula is provable.

This basically concludes the Divide-Part of this algorithm. The only thing left to done in the Conquer step is to check each of these formula. In the case of Justification Logic it means they need to be looked up in the \emph{constant specification}.


\subsection{Atomize}
In this section \emph{formula} usually refers only to the justification term of the formula and is used as a synonym. If it should be understood differently it will be stated so explicitly.
\begin{definition}[atomized]
	A formula or term is called \textbf{atomized} if it fulfills the following conditions:
	\begin{itemize}
		\item The term contains no Sum-Operations.
		\item A Bang-Operation can neither be the top operation of a term nor be the left operant of a Multiplication-Operation.
	\end{itemize}	
\end{definition}
To make the content presented here more understandable the following example will illustrate the steps taken.\footnote{It is on purpose that the \emph{justification term} is by far more complicated than statement $b:F$ that follows the \emph{justification term}. As far as this algorithm goes the complexity of the statement is of no further consequence and thus is kept as simple as possible to allow a easier overview.}

\subsubsection{Sumsplit}
\label{sumsplit}
From the XX Rule of Justification Logic it follows that checking for provability in a formula where the top operation is a sum is equal to checking either operant of the sum and if any of it is provable so is the original formula.

\begin{equation}\label{ss1}
\begin{split}
	(s+t):F \\
	& \Rightarrow s:F \lor t:F
\end{split}
\end{equation}

%!TEX root = ../bachelors_thesis.tex
\begin{figure}[H]
\begin{center}
\begin{tikzpicture}[level distance=1cm,
  level 1/.style={sibling distance=1.2cm}]
	\node [arn_n]{:}
	  child {node [arn_r]{\(+ \)}
	  	child {node [arn_w] {$s$}}
	  	child {node [arn_w] {$t$}}
	  }
	  child {node [arn_w]{$F$}};
\end{tikzpicture}
\hspace{2cm}
\begin{tikzpicture}[level distance=1cm,
  level 1/.style={sibling distance=1.2cm}]
	\node [arn_n]{:}
	  child {node [arn_w]{\(s\)}}
	  child {node [arn_w]{$F$}};
\end{tikzpicture}
\begin{tikzpicture}[level distance=1cm,
  level 1/.style={sibling distance=1.2cm}]
	\node [arn_n]{:}
	  child {node [arn_w]{\(t \)}}
	  child {node [arn_w]{$F$}};
\end{tikzpicture}
\caption{Example of a simple sumsplit.}
\end{center}
\end{figure}


This is of course also true for formulas where Sum is not the top operation. Here $x$ denotes a arbitrary \emph{justification term}.

\begin{equation}\label{ss2}
\begin{split}
	(r*(s+t)):F  \\
	& \Rightarrow r: x \rightarrow F \land (s+t): x \\
	& \Rightarrow ( r: x \rightarrow F \land s: x ) \lor ( r: x \rightarrow F \land t: x )
\end{split}
\end{equation}

%!TEX root = ../bachelors_thesis.tex
\begin{figure}[H]
\begin{center}
\begin{tikzpicture}[level distance=1cm,
  level 1/.style={sibling distance=2cm},
  level 1/.style={sibling distance=1.2cm}]
	\node [arn_n]{:}
	  child {node [arn_w]{\(\cdot \)}
	  	child {node [arn_w] {$r$}}
	  	child {node [arn_w] {$+$}
		  	child {node [arn_w] {$s$}}
		  	child {node [arn_w] {$t$}}	  		
	  	}
	  }
	  child {node [arn_w]{$F$}};
\end{tikzpicture}
\hspace{1.5cm}
\begin{tikzpicture}[level distance=1cm,
  level 1/.style={sibling distance=1.2cm}]
	\node [arn_n]{:}
	  child {node [arn_w]{\(\cdot \)}
	  	child {node [arn_w] {$r$}}
	  	child {node [arn_w] {$s$}}
	  }
	  child {node [arn_w]{$F$}};
\end{tikzpicture}
\begin{tikzpicture}[level distance=1cm,
  level 1/.style={sibling distance=1.2cm}]
	\node [arn_n]{:}
	  child {node [arn_w]{\(\cdot \)}
	  	child {node [arn_w] {$r$}}
	  	child {node [arn_w] {$t$}}
	  }
	  child {node [arn_w]{$F$}};
\end{tikzpicture}
\end{center}
\caption{Example of a sumsplit where the sum is not the top operation.}
\end{figure}

\subsubsection{Simplify Bang}
In this step the aim is to get rid of any Bang-Operator that is the first operation of a formula. Either the Bang can be removed and the formula simplified or else the formula is not provable at all and can be discarded.

Derived from the XX Rule we get the following:

\begin{equation}\label{sb}
\begin{split}
	!t:(t:F)  \\
	& \Rightarrow t: F
\end{split}
\end{equation}

%!TEX root = ../bachelors_thesis.tex
\begin{center}
\begin{tikzpicture}[level distance=1cm,
  level 1/.style={sibling distance=3cm},
  level 2/.style={sibling distance=1.2cm}]
	\node [arn_n]{:}
	  child {node [arn_w]{\(! \)}
	    child[right] {node [subtree] {$t$}}
	    }
	  child {node [arn_w]{:}
  		child {node [subtree] {$t$}}
  		child {node [arn_w] {$F$}}
  		};
\end{tikzpicture}
$\Longrightarrow$
\begin{tikzpicture}[level distance=1cm,
  level 1/.style={sibling distance=1.2cm}]
	\node [arn_n]{:}
	  child {node [subtree]{\(t \)}}
	  child {node [arn_w]{$F$}};
\end{tikzpicture}
\end{center}

Speaking in the manner of a Syntax Tree it needs to be checked, if the child of the Bang-Operation is identical with the left child of the right child of the root. In that case the formula can be simplified to right child of the root only. Else there is no way to resolve the Bang-Operation.

\subsubsection{Remove Bad Bang}
This last step in atomizing the formula proved to be on of the hardest to realize. Only countless examples support the claim that the Bang-Operation must not be the direct left child of a Multiplication-Operation. In coming to that conclusion it has been helpful that no Sum-Operation could make the situation more complex. Because of this and also the fact that a Bang-Operation is never the top operation in a formula it is guarantied that a Bang-Operation must be either a right child or a left child of a Multiplication-Operation.\todoFormat{Operation? Operator?}

\begin{assertion}[Tree Version]
A Bang-Operator that is the direct left child of a Multiplication-Operator causes the whole term to be invalid (unprovable), given that the term is without Sum-Operators and no Bang-Operator at the top. \todoQ{Top-Operation? is it clear that I don't mean ':'?}
\end{assertion}

\begin{equation}\label{bb}
\begin{split}
	(!s*t):F  \\
	& \Rightarrow \exists x : \quad !s: x \rightarrow F \land t: x\\
	& \Rightarrow \exists x,y : \quad !s: x \rightarrow F = !s:(s:y)
\end{split}
\end{equation}

The last line gives a contradiction since there is no possible $x$ or $y$ such that would fulfill the condition of $x \rightarrow F = s:y$.

%!TEX root = ../bachelors_thesis.tex
\begin{figure}[H]
\begin{center}
\begin{tikzpicture}[level distance=1cm,
  level 1/.style={sibling distance=3cm},
  level 2/.style={sibling distance=1.2cm}]
	\node [arn_n]{$:$}
	  child {node [arn_w]{\(\cdot \)}
	    child {node [arn_r] {$!$}
	    		child[right] {node [arn_w] {$t$}}
	    	}
	    child {node [arn_w] {$s$}}
	    }
	  child {node [arn_w]{$F$}};
\end{tikzpicture}
\hspace{2cm}
\begin{tikzpicture}[level distance=1cm,
  level 1/.style={sibling distance=3cm},
  level 2/.style={sibling distance=1.2cm}]
	\node [arn_n]{$:$}
	  child {node [arn_w]{\(\cdot \)}
	    child {node [arn_w] {$s$}}
	    child {node [arn_w] {$!$}
	    		child[right] {node [arn_w] {$t$}}
	    	}
	    }
	  child {node [arn_w]{$F$}};
\end{tikzpicture}
\end{center}
\caption{The left tree shows an introspection that gives a contradiction, while the right tree is valid.}
\end{figure}

\par
This concludes the \emph{atomization} of one formula to many simple formulas which can be checked for provability individually. A formula now consists only of Multiplication-Operations and valid Bang-Operations. The next chapter will show how one atomized formula can be checked for provability.

