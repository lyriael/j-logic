%!TEX root = ../../bachelors_thesis.tex
\section{Divide}
The motivation behind the divide-step existed already long before the actual idea of the Divide and Conquer approach. Given a formula there would be no way to know what kind of formula it was or more precise: what operations were to be found within the \emph{justification term}. The original goal was to find a way to restructure any given formula so that handling it would always need the same steps and not depend to much on what the formula looks exactly. I was looking for something like the $CNF$\footnote{\emph{conjunctive normal form}, a conjunction of clauses, where a clause is a disjunction of literals} and use it in a similar way as $CNF$ is used in $PSC$\footnote{Proof Search Calculus}


\subsection{Atomize}
\subsubsection{Sumsplit}
\subsubsection{Simplify Bang}
\subsubsection{Remove Bad Bang}
\subsection{Get Must}
