%!TEX root = ../bachelors_thesis.tex
Looking back there are many things I would have done differently. There is for example a python module called \emph{Pyparser}\footnote{\href{https://pyparsing.wikispaces.com/}{https://pyparsing.wikispaces.com/}}~\todo{how to format correctly?}. It allows to create and parse simple grammars and it would have served me to get rid of some of the hard-coded string comparison. An other point I often reflected on it the choice of language. There might have more suitable programming languages for this task then python for example a functional language such as haskell. 

On the other hand there are a lot of things that could be done to further improve this implementation. For user to profit from this implementation it would be advantageous to have a (simple) user interface and even more importantly it would be necessary to implement handling of all formulas and not just such implications. A natural first step is adding negation to archive completeness of the logic structure.

A last point that should be considered is the possible optimization of the implementation of this algorithm. As mentioned already in the introduction no special care has been given to consider the tijme efficiency and the algorithm has not been tested for extremely formulas which might cause the implementation to break. 