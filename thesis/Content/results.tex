%!TEX root = ../bachelors_thesis.tex
Looking back there are many things I could have done differently. For example, there is a python module called \emph{Pyparser}\cite{pypar}. It allows the creation and parsing of simple grammars. It would have served me to get rid of some of the hard-coded string comparison. An other point I often reflected on is the choice of language. There might have been more suitable programming languages for this task than Python, for example a functional language like Haskell. 

On the other hand there are a lot of things that could be done to further improve this implementation. It would be advantageous for users to have a user interface. More importantly, it would be benficial if the algorithm didn't just handle implications but other common operators as well. A natural first step is adding negation to achieve completeness.

A last point that should be considered is the possible optimization of the implementation of this algorithm. As mentioned before, no special care has been taken to consider the efficiency. Also the algorithm has not been tested for extremely long input formulas which might cause the implementation to break. 