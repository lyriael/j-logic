%!TEX root = ../bachelors_thesis.tex
\section{Initialization}
In this chapter I would like to walk through one example and covering as many special cases as possible. As such, the justification term we will look at is rather complicated. But this example will also show how nicely this will be broken down in more simpler formulas.

\begin{equation}\label{eq:f}
f = (((!(a+c))+((a+(!a))*(b*(!c)))):(c:F))
\end{equation}


The cs-list used for this example will only be relevant later on but still be presented here as reference:


\begin{equation}\label{cs}
\begin{split}
	cs = \{\\
	& a: [(H \rightarrow (c:F)), ((E \rightarrow (c:D)) \rightarrow (c:F)), (E \rightarrow (c:D))],\\
	& b: [((c:F) \rightarrow H), ((c:D) \rightarrow (a:F)), ((H \rightarrow G) \rightarrow H), (Y_1 \rightarrow (Y_2 \rightarrow Y_1))],\\
	& c: [(c:F), G, D, (G \rightarrow F)] \\
	\}
\end{split}
\end{equation}

The data presented here is in the same form as it would be entered into the program. Therefore the cs-list is rather a \emph{python} dictionary than a simple tuple-list and there are more brackets explicitly written then required by convention.


\section{Walking in Trees: Atomize}

The given formula $f$ will be transformed into a syntax tree using \emph{parse\_formula} of \emph{Tree}. If the formula is provided when the \emph{ProofSearch} object is initialized the formula will automatically be atomized without having to call this method separately.

\begin{equation*}
	(((!(a
    \tikz[baseline]{\node[fill=red!20,anchor=base] (t1){$+$};} c))
    \tikz[baseline]{\node[fill=red!20,anchor=base] (t2){$+$};} ((a
    \tikz[baseline]{\node[fill=red!20,anchor=base] (t3){$+$};} (!a))*(b*(!c)))):(c:F))
\end{equation*}

%!TEX root = ../../bachelors_thesis.tex
\begin{figure}[H]
\caption{Syntax tree of given formula $f$ before it is atomized.}
\begin{center}
\begin{tikzpicture}[level distance=1.2cm,
  level 1/.style={sibling distance=5cm},
  level 2/.style={sibling distance=3.5cm}, 
  level 3/.style={sibling distance=2cm},
  level 4/.style={sibling distance=1cm},
  level 5/.style={sibling distance=1.5cm}, thick,scale=0.9, every node/.style={scale=0.9}]
	\node [arn_n]{$:$}
	  child {node [arn_r]{\(+ \)}
	    child {node [arn_w] {$!$}
	    	child[right] {node [arn_r] {$+$}
	    		child {node [arn_w] {$a$}}
	    		child {node [arn_w] {$c$}}
	    		}
	    	}
	    child {node [arn_w] {$*$}
	    	child {node [arn_r] {$+$}
	    		child {node [arn_w] {$a$}}
	    		child {node [arn_w] {$!$}
	    			child[right] {node [arn_w] {$a$}}
	    			}
	    		}
	    	child {node [arn_w] {$*$}
	    		child {node [arn_w] {$b$}}
	    		child {node [arn_w] {$!$}
	    			child[right] {node [arn_w] {$c$}}
	    			}
	    		}
	    	}
	    }
	  child {node [arn_w]{$:$}
	  	child {node [arn_w] {$c$}}
	  	child {node [arn_w] {$F$}}
	  };
\end{tikzpicture}
\end{center}
\end{figure}


The \emph{sum\_split} from \emph{Tree} will give us the following terms in form of a list. 

\begin{equation}\label{eq:sp_i}	
	((!a):(c:F))
\end{equation}
\begin{equation}\label{eq:sp_ii}	
	((!c):(c:F))
\end{equation}	
\begin{equation}\label{eq:sp_iii}	
	((a *(b* (! c))):(c:F))											
\end{equation}
\begin{equation}\label{eq:sp_iv}	
	(((! a)*(b* (! c))):(c:F))												
\end{equation}

\subsection{Bangs}
Looking at each of the terms individually we will now further look at them to discard any that have a \emph{bad bang}, meaning a bang that is the left child of a multiplication or if there are terms with bang which can be simplified.

\subsubsection[First term]{Term \ref{eq:sp_i}}
In this term we find a bang which is valid, since it is not a left child of a multiplication, but trying to simplify the term shows us that it cannot be resolved thus letting us discard is term.

\begin{equation*}
	((\tikz[baseline]{\node[fill=red!20,anchor=base] (t1){$!$};}a):(c:F)) 
\end{equation*}

\subsubsection[Second term]{Term \ref{eq:sp_ii}}
As before the bang within the term is valid but in contrast to the previous example the term here can be simplified, giving us our first \emph{atom} for formula $f.$

\begin{equation}\label{eq:a_1}
	\begin{split}
	((\tikz[baseline]{\node[fill=red!20,anchor=base] (t1){$!$};} c):(c:F)) & \Rightarrow \\
	& a_1 := (c:F)
	\end{split}
\end{equation}

\subsubsection[third term]{Term \ref{eq:sp_iii}}
In this term we find the bang operation neither a left child of a multiplication nor as top operation of the proof term and thus there is nothing to do.

\begin{equation}\label{eq:a_2}
	\begin{split}
	((a *(b* (\tikz[baseline]{\node[fill=red!20,anchor=base] (t1){$!$};} c))):(c:F))	 & \Rightarrow \\
	& a_2 := ((a *(b* (! c))):(c:F))
	\end{split}
\end{equation}

\subsubsection[Fourths term]{Term \ref{eq:sp_iv}}
Finally this term has two bangs of which the first is the left child of a multiplication and thus makes the term invalid. The second bang would be valid, but the first term causes the whole term to be discarded.

\begin{equation*}
	(((
	\tikz[baseline]{\node[fill=red!20,anchor=base] (t1){$!$};} a)*(b* (
	\tikz[baseline]{\node[fill=red!20,anchor=base] (t1){$!$};} c))):(c:F))
\end{equation*}

\bigskip
This completes the \emph{atomize} step for the formula $f$ giving us two \emph{atoms}. Showing that at least one of those is provable is enough to show that $f$ is provable. 

\section{Looking up and merging}

\begin{equation*}
		f = (((!(a+c))+((a+(!a))*(b*(!c)))):(c:F)) 
		\tag{\ref{eq:f}}  
\end{equation*}
For our formula $f$ we have found the two atoms \ref{eq:a_1} and \ref{eq:a_2}. The next steps will be determining the \emph{musts} if needed, matching them against the cs-list and finally merge the possible configurations together to determine if one of the musts is provable.


\begin{equation*}
		a_1 = (c:F) 
		\tag{\ref{eq:a_1}}
\end{equation*}
\begin{equation*}		
		a_2 = ((a *(b* (! c))):(c:F)) 
		\tag{\ref{eq:a_2}}
\end{equation*}

\subsection{Musts}
\subsubsection[First atom]{Atom \ref{eq:a_1}}
Since $a_1$ consists already only of one proof constant with the correspond term to it there is nothing further to to here.
\begin{equation}
	a_1: \quad musts = [(c, F)]
\end{equation}

\subsubsection[First atom]{Atom \ref{eq:a_2}}
For $a_2$ we need to take the proof term apart bit by bit. The first operation we will take apart is a multiplication. Extracting proof constants from a multiplication proof term will always us give a \emph{X-wild}. Whenever a new \emph{X-wilds} appears the $i$ of $X_i$ will simply be increased by 1.

\begin{equation*}\label{eq:musts1_a_2}
	\begin{split}
		((a *(b* (! c))):(c:F)) & \Rightarrow \\
		& a : (X_1 \rightarrow (c:F)) \\
		& (b*(! c)): X_1
	\end{split}	
\end{equation*}

The proof constant $a$ has been isolated but $(b*(! c))$ still needs further taking apart. We repeat the step from above and introduce yet another \emph{X-wild}.

\begin{equation*}
	\begin{split}
	(b*(! c)): X_1 & \Rightarrow \\
	& b : (X_2 \rightarrow X_1) \\
	& (! c) : X_2
	\end{split}	
\end{equation*}

Now $b$ has been isolated as well, leaving only $(! c)$. Having a bang in a situation like this results in a new \emph{X-wild} in combination with the proof term which will replace a previous \emph{X-wild}.

\begin{equation*}
	\begin{split}
		(! c) : X_2 & \Rightarrow \\
		& X_2 = (c:X_3)
	\end{split}	
\end{equation*}

This finally gives us all the \emph{musts} for $a_2$. As can be seen belove the \emph{X-wild} $X_2$ has been replaced by $(c:X_3)$.
\begin{equation}\label{eq:a2_musts}
	a_2: \quad musts = [(a, (X_1 \rightarrow (c:F))), (b, ((c:X_3) \rightarrow X_1)), (c, X_3)]
\end{equation}

It should be noted here that a proof constant may be in more than one of the \emph{musts} for one \emph{atom}. 

\subsection{Configurations and Conditions}
Now that we have all proof constants for each \emph{atom} isolated we can try to match them with the entries found in the cs-list.

\begin{equation*}
\tag{\ref{cs}}
\begin{split}
	cs = \{\\
	& a: [(H \rightarrow (c:F)), ((E \rightarrow (c:D)) \rightarrow (c:F)), (E \rightarrow (c:D))],\\
	& b: [((c:F) \rightarrow H), ((c:D) \rightarrow (a:F)), ((H \rightarrow G) \rightarrow H), (Y_1 \rightarrow (Y_2 \rightarrow Y_1))],\\
	& c: [(c:F), G, D, (G \rightarrow F)] \\
	\}
\end{split}
\end{equation*}

\subsubsection[Config for a1]{Configs for musts of atom $a_1$}
The \emph{atom} $a_1$ actually has only one rather simple \emph{must}: $(c,F)$. As it can easily be seen there is no such match with the cs-list for the proof constant $c$.

If we in fact would have found a valid match we would have also shown that the formula $f$ is provable. Also the algorithm would usually stop as soon as one must is provable unless explicitly told so.

\subsubsection[Config for a2]{Configs for musts of atom $a_1$}

\begin{equation*}
	a_2: \quad musts = [(a, (X_1 \rightarrow (c:F))), (b, ((c:X_3) \rightarrow X_1)), (c, X_3)]
	\tag{\ref{eq:a2_musts}}
\end{equation*}

Matching the \emph{musts} with the entries in cs gives us the following table. Since the \emph{X-wild} $X_2$ in never used it will be omitted here. In the implementation it would just be an additional empty column.

\begin{table}[H]
\centering
\begin{tabular}{l c c r}
		& $X_1$ & $X_3$ & condition \\
	\hline 
	$a$	& $H$ & - &  \\
		& $(E \rightarrow (c:D))$ & - & \\
	\hline
	$b$ & $H$ & $F$ & \\
		& $(a:F)$ & $D$ & \\
		& $H$ & $(H \rightarrow G)$ & \\
		& - & - & $(X_1, (Y_2 \rightarrow (c:X_3)))$\\
	\hline
	$c$	& - & $(c:F)$ & \\
		& - & $G$ & \\
		& - & $D$ & \\
		& - & $(G \rightarrow F)$ &\\
	\hline
\end{tabular}
\caption{Config table for $a_2$}\label{config}
\end{table}

All match except the match with the \emph{Y-wilds} are straight forward and need no further explanations, and the \emph{Y-wilds} match shall be explained in more detail here. 

%!TEX root = ../../bachelors_thesis.tex
\begin{figure}[H]
	\centering
	% (Y_1 -> (Y_2 -> Y_1))
	\begin{tikzpicture}[level distance=1.2cm,
	  level 1/.style={sibling distance=2.5cm},
	  level 2/.style={sibling distance=1.5cm},
	  level 3/.style={sibling distance=0.8cm}, 
	  thick,scale=0.9, every node/.style={scale=0.9}]
		\node [arn_w]{$\rightarrow$}
		  child {node [arn_r]{$Y_1$}}
		  child {node [arn_w]{$\rightarrow$}
		  	child {node [arn_w] {$Y_2$}}
		  	child {node [arn_w] {$Y_1$}}
		  };
	\end{tikzpicture}
	% (X_1 -> (c:F))
	\begin{tikzpicture}[level distance=1.2cm,
	  level 1/.style={sibling distance=2.5cm},
	  level 2/.style={sibling distance=1.5cm},
	  level 3/.style={sibling distance=0.8cm}, 
	  thick,scale=0.9, every node/.style={scale=0.9}]
		\node [arn_w]{$\rightarrow$}
		  child {node [arn_r]{$:$}
		  	child {node [arn_r] {$c$}}
		  	child {node [arn_r] {$X_3$}}
		  }
		  child {node [arn_w]{$X_1$}};
	\end{tikzpicture}
\captionof{figure}{Comparing a term from cs-list with Y-wilds with a must term.}
\label{figure_compare1}
\end{figure}


When the \emph{compare} method from \emph{Tree} encounters a node containing a \emph{Y-wild} it will replace it and all other occurrences with the same \emph{Y-wild} in its tree with whatever it finds in the node of the comparing tree. In this case it will replace all $Y_1$ with $(c:X_3)$. 

%!TEX root = ../../bachelors_thesis.tex
\begin{figure}[H]
	\centering
	% (X_1 -> (Y_2 -> X_1))
	\begin{tikzpicture}[level distance=1.2cm,
	  level 1/.style={sibling distance=2.5cm},
	  level 2/.style={sibling distance=1.5cm},
	  level 3/.style={sibling distance=1.2cm}, 
	  thick,scale=0.9, every node/.style={scale=0.9}]
		\node [arn_w]{$\rightarrow$}
		  child {node [arn_w]{$:$}
		  	child {node [arn_w] {$c$}}
		  	child {node [arn_w] {$X_3$}}
		  	}
		  child {node [arn_r]{$\rightarrow$}
		  	child {node [arn_r] {$Y_2$}}
		  	child {node [arn_r]{$:$}
		  		child {node [arn_r] {$c$}}
		  		child {node [arn_r] {$X_3$}}
		  	}
		  };
	\end{tikzpicture}
	% (X_1 -> (c:X_3))
	\begin{tikzpicture}[level distance=1.2cm,
	  level 1/.style={sibling distance=2.5cm},
	  level 2/.style={sibling distance=1.5cm},
	  level 3/.style={sibling distance=1.2cm}, 
	  thick,scale=0.9, every node/.style={scale=0.9}]
		\node [arn_w]{$\rightarrow$}
		  child {node [arn_w]{$:$}
		  	child {node [arn_w] {$c$}}
		  	child {node [arn_w] {$X_3$}}
		  }
		  child {node [arn_r]{$X_1$}};
	\end{tikzpicture}
\captionof{figure}{Updated cs-term (copy) with resulting condition.}
\label{figure_compare2}
\end{figure}

When the methods now compares the right child of the root if finds a \emph{X-wild} in the \emph{must} term, it now checks if there are \emph{X-wilds} or \emph{Y-wilds} present in the subtree of the Node from the cs term. If that is the case, as it is here, the subtree be be a condition to the $X_i$. If the term would have consisted of operators and constants only the subtree would have been written directly into the configuration table. But as that is not the case here, we end up with a empty row in the config table and the condition $(X_1, (Y_2 \rightarrow (c:X_3)))$.

\subsection{Merging of the Config Table}

As we have only one atom left that might still be provable we have to merge the different sections from the config table \ref{config} to find at least one configuration for all \emph{X-wilds} that is contradiction free for all proof constants. The \emph{merge}-step in the algorithm merges two sections and adds to the result the next table, so in this example we will first merge $a$ and $b$ and to the result of that merge $c$.

\begin{table}[H]
\centering
\begin{tabular}{l c c r}
		& $X_1$ & $X_3$ & condition \\
	\hline 
	$a$	& $H$ & - &  \\
		& $(E \rightarrow (c:D))$ & - & \\
	\hline
	$b$ & $H$ & $F$ & \\
		& $(a:F)$ & $D$ & \\
		& $H$ & $(H \rightarrow G)$ & \\
		& - & - & $(X_1, (Y_2 \rightarrow (c:X_3)))$\\
	\hline
	\hline
	$a \cap b$	& $H$ & $F$ & \\
		& $H$ & $(H \rightarrow G)$ & \\
		& $(E \rightarrow (c:D))$ & $D$ & \\
	\hline
\end{tabular}
\caption{Merge of section $a$ and $b$}\label{merge_a_b}
\end{table}

The first two merges are obvious, there are however some comments on the last one. As $E \rightarrow (c:D)$ for fits perfect the condition $Y_2 \rightarrow (c:X_3)$ for $X_1$ giving us $X_3 := D$, the second line from $a$ can be merged with the third from $b$. In this case the condition can be dropped because it has fully been satisfied, but in other cases, especially if there are more \emph{X-wilds} it is possible that a condition still persists after a merge. Also we found that $Y_2$ must be $E$ which would matter, if there were other conditions on the same configuration line that also contained a $Y_2$ (rather unlikely but very possible).

Now we will merge the section from $c$ to our current result of table \ref{merge_a_b}.

\begin{table}[H]
\centering
\begin{tabular}{l c c r}
		& $X_1$ & $X_3$ & condition \\
	\hline
	$a \cap b$	& $H$ & $F$ & \\
		& $H$ & $(H \rightarrow G)$ & \\
		& $(E \rightarrow (c:D))$ & $D$ & \\
	\hline
	$c$	& - & $(c:F)$ & \\
		& - & $G$ & \\
		& - & $D$ & \\
		& - & $(G \rightarrow F)$ &\\
	\hline
	\hline
	$(a \cap b) \cap c$ & $(E \rightarrow (c:D))$ & $D$ \\
	\hline

\end{tabular}
\caption{Merge of section $c$ with the previously merged $a$ and $b$.}\label{merge_a_b_c}
\end{table}

\section{The Final Result}
As we have seen in table \ref{merge_a_b_c} we did find a configuration that fits the \emph{X-wilds} and thus the original formula $f$ is provable. I want to show here what this configuration actually means and what it has to do with the provability of $f$. To do this we will actually go through the algorithm backwards but having already the solution at hand.

What we gained from the last step is a configuration for the \emph{X-wilds}.
\begin{equation}
	\begin{split}
	X_1 & = (E \rightarrow (c:D))\\
	X_2 & = (c:X_3) = (c:D) \\
	X_3 & = D 
	\end{split}	
\end{equation}

Knowing these we can use them and replace them in the musts of the atom $a_2$. That way we get what we where looking for in the cs-list.

\begin{equation*}
\begin{split}
	musts \medspace a_2: \quad  [&(a, ((E \rightarrow (c:D)) \rightarrow (c:F))), \\
			& (b, ((c:D) \rightarrow (E \rightarrow (c:D)))), \\
			& (c, D)]
\end{split}
\tag{\ref{eq:a2_musts}}
\end{equation*}

\begin{equation*}
\tag{\ref{cs}}
\begin{split}
	cs = \{& a: [(H \rightarrow (c:F)), ((E \rightarrow (c:D)) \rightarrow (c:F)), (E \rightarrow (c:D))],\\
	& b: [((c:F) \rightarrow H), ((c:D) \rightarrow (a:F)), ((H \rightarrow G) \rightarrow H), (Y_1 \rightarrow (Y_2 \rightarrow Y_1))],\\
	& c: [(c:F), G, D, (G \rightarrow F)] \}
\end{split}
\end{equation*}

As expected we find all the terms from \emph{must} precisely like that in the cs-list. Also we can reconstruct $a_2$ with the \emph{musts} given above:

\begin{align*}
	c:D & \Rightarrow (!c):(c:D)\\
	b:((c:D) \rightarrow (E \rightarrow (c:D)))), (!c):(c:D) & \Rightarrow ((b*(!c)):(E \rightarrow (c:D)))\\
	a:((E \rightarrow (c:D)) \rightarrow (c:F))), (b*(!c)):(E \rightarrow (c:D)) & \Rightarrow (a *(b* (! c)):(c:F)) 
\end{align*}

\begin{equation*}		
	a_2 = ((a *(b* (! c))):(c:F)) 
	\tag{\ref{eq:a_2}}
\end{equation*}

Reconstructing the whole formula $f$ instead of just the atom follows the same approach as shown above but it will become very verbose and it is very little gained if one is not in doubt about the \emph{atomization}.

\begin{equation*}
	f = (((!(a+c))+((a+(!a))*(b*(!c)))):(c:F))
	\tag{\ref{eq:f}}
\end{equation*}

%!TEX root = ../../bachelors_thesis.tex
\begin{figure}[H]
\begin{center}
\begin{tikzpicture}[level distance=1.2cm,
  level 1/.style={sibling distance=5cm},
  level 2/.style={sibling distance=3.5cm}, 
  level 3/.style={sibling distance=2cm},
  level 4/.style={sibling distance=1cm},
  level 5/.style={sibling distance=1.5cm}, thick,scale=0.9, every node/.style={scale=0.9}]
	\node [arn_n]{$:$}
	  child {node [arn_r]{\(+ \)}
	    child {node [arn_w] {$!$}
	    	child[right] {node [arn_w] {$+$}
	    		child {node [arn_w] {$a$}}
	    		child {node [arn_w] {$c$}}
	    		}
	    	}
	    child {node [arn_r] {$*$}
	    	child {node [arn_r] {$+$}
	    		child {node [arn_r] {$a$}}
	    		child {node [arn_w] {$!$}
	    			child[right] {node [arn_w] {$a$}}
	    			}
	    		}
	    	child {node [arn_r] {$*$}
	    		child {node [arn_r] {$b$}}
	    		child {node [arn_r] {$!$}
	    			child[right] {node [arn_r] {$c$}}
	    			}
	    		}
	    	}
	    }
	  child {node [arn_r]{$:$}
	  	child {node [arn_r] {$c$}}
	  	child {node [arn_r] {$F$}}
	  };
\end{tikzpicture}
\caption{Syntax tree of formula $f$ with atom $a_2$ hightlighted.}
\end{center}
\end{figure}



This concludes this chapter where I tried to show as much as possible with an example that is as short and simple as possible and still fits the purpose. 
