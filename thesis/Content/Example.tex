%!TEX root = ../bachelors_thesis.tex
\section{Initialization}
In this chapter I would like to walk through one example and covering as many special cases as possible. As such, the justification term we will look at is rather complicated. But this example will also show how nicely this will be broken down in more simpler formulas.

\begin{equation}\label{eq:f}
f = (((!(a+c))+((a+(!a))*(b*(!c)))):(c:F))
\end{equation}


The cs-list used for this example will only be relevant later on but still be presented here as reference:


\begin{equation}\label{cs}
\begin{split}
	cs = \{\\
	& a: [(H \rightarrow (c:F)), ((E \rightarrow (c:D)) \rightarrow (c:F)), (E \rightarrow (c:D))],\\
	& b: [((c:F) \rightarrow H), ((c:D) \rightarrow (a:F)), ((H \rightarrow G) \rightarrow H), (Y_1 \rightarrow (Y_2 \rightarrow Y_1))],\\
	& c: [(c:F), G, D, (G \rightarrow F)] \\
	\}
\end{split}
\end{equation}

The data presented here is in the same form as it would be entered into the program. Therefore the cs-list is rather a \emph{python} dictionary than a simple tuple-list and there are more brackets explicitly written then required by convention.


\section{Walking in Trees: Atomize}

The given formula $f$ will be transformed into a syntax tree using \emph{parse\_formula} of \emph{Tree}. If the formula is provided when the \emph{ProofSearch} object is initialized the formula will automatically be atomized without having to call this method separately.

\begin{equation*}
	(((!(a
    \tikz[baseline]{\node[fill=red!20,anchor=base] (t1){$+$};} c))
    \tikz[baseline]{\node[fill=red!20,anchor=base] (t2){$+$};} ((a
    \tikz[baseline]{\node[fill=red!20,anchor=base] (t3){$+$};} (!a))*(b*(!c)))):(c:F))
\end{equation*}

%!TEX root = ../../bachelors_thesis.tex
\begin{figure}[H]
\begin{center}
\begin{tikzpicture}[level distance=1.2cm,
  level 1/.style={sibling distance=5cm},
  level 2/.style={sibling distance=3.5cm}, 
  level 3/.style={sibling distance=2cm},
  level 4/.style={sibling distance=1cm},
  level 5/.style={sibling distance=1.5cm}, thick,scale=0.9, every node/.style={scale=0.9}]
	\node [arn_n]{$:$}
	  child {node [arn_r]{\(+ \)}
	    child {node [arn_w] {$!$}
	    	child[right] {node [arn_r] {$+$}
	    		child {node [arn_w] {$a$}}
	    		child {node [arn_w] {$c$}}
	    		}
	    	}
	    child {node [arn_w] {$*$}
	    	child {node [arn_r] {$+$}
	    		child {node [arn_w] {$a$}}
	    		child {node [arn_w] {$!$}
	    			child[right] {node [arn_w] {$a$}}
	    			}
	    		}
	    	child {node [arn_w] {$*$}
	    		child {node [arn_w] {$b$}}
	    		child {node [arn_w] {$!$}
	    			child[right] {node [arn_w] {$c$}}
	    			}
	    		}
	    	}
	    }
	  child {node [arn_w]{$:$}
	  	child {node [arn_w] {$c$}}
	  	child {node [arn_w] {$F$}}
	  };
\end{tikzpicture}
\caption{Syntax tree of given formula $f$ before it is atomized.}
\end{center}
\end{figure}


The \emph{sum\_split} from \emph{Tree} will give us the following terms in form of a list. 

\begin{equation}\label{eq:sp_i}	
	((!a):(c:F))
\end{equation}
\begin{equation}\label{eq:sp_ii}	
	((!c):(c:F))
\end{equation}	
\begin{equation}\label{eq:sp_iii}	
	((a *(b* (! c))):(c:F))											
\end{equation}
\begin{equation}\label{eq:sp_iv}	
	(((! a)*(b* (! c))):(c:F))												
\end{equation}

\subsection{Bangs}
Looking at each of the terms individually we will now further look at them to discard any that have a \emph{bad bang}, meaning a bang that is the left child of a multiplication or if there are terms with bang which can be simplified.

\subsubsection[First term]{Term \ref{eq:sp_i}}
In this term we find a bang which is valid, since it is not a left child of a multiplication, but trying to simplify the term shows us that it cannot be resolved thus letting us discard is term.

\begin{equation*}
	((\tikz[baseline]{\node[fill=red!20,anchor=base] (t1){$!$};}a):(c:F)) 
\end{equation*}

\subsubsection[Second term]{Term \ref{eq:sp_ii}}
As before the bang within the term is valid but in contrast to the previous example the term here can be simplified, giving us our first \emph{atom} for formula $f.$

\begin{equation}\label{eq:a_1}
	\begin{split}
	((\tikz[baseline]{\node[fill=red!20,anchor=base] (t1){$!$};} c):(c:F)) & \Rightarrow \\
	& a_1 := (c:F)
	\end{split}
\end{equation}

\subsubsection[third term]{Term \ref{eq:sp_iii}}
In this term we find the bang operation neither a left child of a multiplication nor as top operation of the proof term and thus there is nothing to do.

\begin{equation}\label{eq:a_2}
	\begin{split}
	((a *(b* (\tikz[baseline]{\node[fill=red!20,anchor=base] (t1){$!$};} c))):(c:F))	 & \Rightarrow \\
	& a_2 := ((a *(b* (! c))):(c:F))
	\end{split}
\end{equation}

\subsubsection[Fourths term]{Term \ref{eq:sp_iv}}
Finally this term has two bangs of which the first is the left child of a multiplication and thus makes the term invalid. The second bang would be valid, but the first term causes the whole term to be discarded.

\begin{equation*}
	(((
	\tikz[baseline]{\node[fill=red!20,anchor=base] (t1){$!$};} a)*(b* (
	\tikz[baseline]{\node[fill=red!20,anchor=base] (t1){$!$};} c))):(c:F))
\end{equation*}

\bigskip
This completes the \emph{atomize} step for the formula $f$ giving us two \emph{atoms}. Showing that at least one of those is provable is enough to show that $f$ is provable. 

\section{Looking up and merging}

\begin{equation*}
		f = (((!(a+c))+((a+(!a))*(b*(!c)))):(c:F)) 
		\tag{\ref{eq:f}}  
\end{equation*}
For our formula $f$ we have found the two atoms \ref{eq:a_1} and \ref{eq:a_2}. The next steps will be determining the \emph{musts} if needed, matching them against the cs-list and finally merge the possible configurations together to determine if one of the musts is provable.


\begin{equation*}
		a_1 = (c:F) 
		\tag{\ref{eq:a_1}}
\end{equation*}
\begin{equation*}		
		a_2 = ((a *(b* (! c))):(c:F)) 
		\tag{\ref{eq:a_2}}
\end{equation*}

\subsection{Musts}
\subsubsection[First atom]{Atom \ref{eq:a_1}}
Since $a_1$ consists already only of one proof constant with the correspond term to it there is nothing further to to here.
\begin{equation}
	a_1: \quad musts = [(c, F)]
\end{equation}

\subsubsection[First atom]{Atom \ref{eq:a_2}}
For $a_2$ we need to take the proof term apart bit by bit. The first operation we will take apart is a multiplication. Extracting proof constants from a multiplication proof term will always us give a \emph{X-wild}. Whenever a new \emph{X-wilds} appears the $i$ of $X_i$ will simply be increased by 1.

\begin{equation*}\label{eq:musts1_a_2}
	\begin{split}
		((a *(b* (! c))):(c:F)) & \Rightarrow \\
		& a : (X_1 \rightarrow (c:F)) \\
		& (b*(! c)): X_1
	\end{split}	
\end{equation*}

The proof constant $a$ has been isolated but $(b*(! c))$ still needs further taking apart. We repeat the step from above and introduce yet another \emph{X-wild}.

\begin{equation*}
	\begin{split}
	(b*(! c)): X_1 & \Rightarrow \\
	& b : (X_2 \rightarrow X_1) \\
	& (! c) : X_2
	\end{split}	
\end{equation*}

Now $b$ has been isolated as well, leaving only $(! c)$. Having a bang in a situation like this results in a new \emph{X-wild} in combination with the proof term which will replace a previous \emph{X-wild}.

\begin{equation*}
	\begin{split}
		(! c) : X_2 & \Rightarrow \\
		& X_2 = (c:X_3)
	\end{split}	
\end{equation*}

This finally gives us all the \emph{musts} for $a_2$. As can be seen belove the \emph{X-wild} $X_2$ has been replaced by $(c:X_3)$.
\begin{equation}\label{eq:a2_musts}
	a_2: \quad musts = [(a, (X_1 \rightarrow (c:F))), (b, ((c:X_3) \rightarrow X_1)), (c, X_3)]
\end{equation}

It should be noted here that a proof constant may be in more than one of the \emph{musts} for one \emph{atom}. 

\todoWriteMore{Finish Example}

This concludes this chapter where I tried to show as much as possible with an example that is as short and simple as possible and still fits the purpose. 
