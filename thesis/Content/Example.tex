%!TEX root = ../bachelors_thesis.tex
In this chapter I will walk through a example covering a few key cases. The justification term we will look at is rather complicated but the walkthrough will also show how it can be broken down into simpler terms.

\section{Initialization}
The algorithm takes a formula $f$ and a cs-list as input. The data presented here is in the same form as it would be entered into the program. Therefore the cs-list is a \emph{python} dictionary and not a simple list of pairs as given by the definition of justification logic. Also, there are more brackets explicitly written than required by convention to enable parsing.
\begin{align}\label{eq:f}
f = (((!(a+c))+((a+(!a))\cdot (b\cdot (!c)))):(c:F))
\end{align}

\begin{equation}\label{cs}
\begin{split}
	cs = \{& a: [(H \rightarrow (c:F)), ((E \rightarrow (c:D)) \rightarrow (c:F)), (E \rightarrow (c:D))],\\
	& b: [((c:F) \rightarrow H), ((c:D) \rightarrow (a:F)), ((H \rightarrow G) \rightarrow H), (Y_1 \rightarrow (Y_2 \rightarrow Y_1))],\\
	& c: [(c:F), G, D, (G \rightarrow F)]\}
\end{split}
\end{equation}

\section{Walking in Trees: Atomize}

The given formula $f$ is transformed into a syntax tree using \texttt{parse\_formula} of \texttt{Tree}. 

\subsection{Sumsplit}
Our first step is to split our formula for every sum we encounter.
\begin{equation*}
	(((!(a
    \tikz[baseline]{\node[fill=blue!20,anchor=base] (t1){$+$};} c))
    \tikz[baseline]{\node[fill=blue!20,anchor=base] (t2){$+$};} ((a
    \tikz[baseline]{\node[fill=blue!20,anchor=base] (t3){$+$};} (!a))\cdot (b\cdot (!c)))):(c:F))
\end{equation*}

%!TEX root = ../../bachelors_thesis.tex
\begin{figure}[H]
\begin{center}
\begin{tikzpicture}[level distance=1.2cm,
  level 1/.style={sibling distance=5cm},
  level 2/.style={sibling distance=3.5cm}, 
  level 3/.style={sibling distance=2cm},
  level 4/.style={sibling distance=1cm},
  level 5/.style={sibling distance=1.5cm}, thick,scale=0.9, every node/.style={scale=0.9}]
	\node [arn_n]{$:$}
	  child {node [arn_r]{\(+ \)}
	    child {node [arn_w] {$!$}
	    	child[right] {node [arn_r] {$+$}
	    		child {node [arn_w] {$a$}}
	    		child {node [arn_w] {$c$}}
	    		}
	    	}
	    child {node [arn_w] {$*$}
	    	child {node [arn_r] {$+$}
	    		child {node [arn_w] {$a$}}
	    		child {node [arn_w] {$!$}
	    			child[right] {node [arn_w] {$a$}}
	    			}
	    		}
	    	child {node [arn_w] {$*$}
	    		child {node [arn_w] {$b$}}
	    		child {node [arn_w] {$!$}
	    			child[right] {node [arn_w] {$c$}}
	    			}
	    		}
	    	}
	    }
	  child {node [arn_w]{$:$}
	  	child {node [arn_w] {$c$}}
	  	child {node [arn_w] {$F$}}
	  };
\end{tikzpicture}
\caption{Syntax tree of given formula $f$ before it is atomized.}
\end{center}
\end{figure}


The \texttt{sum\_split} from \texttt{Tree} will give us the following terms in form of a list. 

\begin{equation}\label{eq:sp_i}	
	((!a):(c:F))
\end{equation}
\begin{equation}\label{eq:sp_ii}	
	((!c):(c:F))
\end{equation}	
\begin{equation}\label{eq:sp_iii}	
	((a \cdot(b\cdot (! c))):(c:F))											
\end{equation}
\begin{equation}\label{eq:sp_iv}	
	(((! a)\cdot(b\cdot (! c))):(c:F))												
\end{equation}
Some of those will eventually be the atoms for $f$, but before they can become atoms of $f$ they need to be simplified and checked with regard to their introspection operators.

\subsection{Introspections}
If the proof term starts with an introspection operator, there are two possibilities: First, the term could be simplified. Second, the term could be discarded. See the examples below for an illustration of those cases. 

If we find an introspection operator as a child of a multiplication operator, there are again two possibilities: First, if we find the introspection operator to be the right child of the multiplication operator, the term is valid, meaning it contains no inherent contradictions. Second, if we find the introspection operator to be the left child of the multiplication operator, the term is invalid. 

An invalid term is discarded.

\subsubsection[First term]{Term (\ref{eq:sp_ii})}
The introspection operation for this justification term can be simplified, giving us our first \emph{atom} for formula $f.$
\begin{equation}\label{eq:a_1}
	\begin{split}
	((\tikz[baseline]{\node[fill=blue!20,anchor=base] (t1){$!$};} c):(c:F)) \Rightarrow \\
	& a_1 := (c:F)
	\end{split}
\end{equation}

\subsubsection[Second term]{Term (\ref{eq:sp_i})}
In this term we find an introspection operation which when trying to simplify shows us that it cannot be resolved. Thus, it is no atom and we discard this term.
\begin{equation*}
	((\tikz[baseline]{\node[fill=blue!20,anchor=base] (t1){$!$};}a):(c:F))\; \Rightarrow \; \text{\Lightning} 
\end{equation*}


\subsubsection[Third term]{Term (\ref{eq:sp_iii})}
In this term we find that the introspection operator is neither a left child of a multiplication operator nor a top operator of the proof term and thus we have our second \emph{atom}.
\begin{equation}\label{eq:a_2}
	\begin{split}
	((a \cdot(b\cdot (\tikz[baseline]{\node[fill=blue!20,anchor=base] (t1){$!$};} c))):(c:F)) \Rightarrow \\
	& a_2 := ((a \cdot(b\cdot (! c))):(c:F))
	\end{split}
\end{equation}


\subsubsection[Fourth term]{Term (\ref{eq:sp_iv})}
This proof term has two introspection operators of which the first is the left child of a multiplication operator, thus making the term invalid. The second introspection would be valid, but the first term causes the whole proof term to be discarded.
\begin{equation*}
	(((
	\tikz[baseline]{\node[fill=blue!20,anchor=base] (t1){$!$};} a)\cdot(b\cdot (
	\tikz[baseline]{\node[fill=blue!20,anchor=base] (t1){$!$};} c))):(c:F)) \; \Rightarrow \; \text{\Lightning} 
\end{equation*}


\bigskip
This completes the \texttt{atomize} step for the formula $f$ giving us the two atoms $a_1$ and $a_2$. Showing that at least one of those is provable is enough to show that $f$ is provable. 

\section{Getting and Looking up the Musts}
We have found the two atoms $a_1$ and $a_2$ for the formula $f$. The next step determines the \emph{musts}, matching them against the cs-list and finally merge the possible configurations together to determine if one of the musts is provable.

\subsection{Musts}
\subsubsection[First atom]{Atom $a_1$ (\ref{eq:a_1})}
Since $a_1$ consists of only one proof constant already, there is nothing further to to here.
\begin{equation}
	musts \; a_1:\quad  [(c, F)]
\end{equation}
\subsubsection[Second atom]{Atom $a_2$ (\ref{eq:a_2})}
For $a_2$ we need to take the proof term apart bit by bit. Extracting proof constants from a proof term gives us an \emph{X} variable for each application operator. 
\begin{align}\label{eq:musts1_a_2}
		((a \cdot(b\cdot (! c))):(c:F)) \Rightarrow \nonumber\\
		 & a : \; (X_1 \rightarrow (c:F)) \nonumber\\
		 & (b\cdot(! c)): \; X_1
\end{align}

The proof constant $a$ has been isolated. We repeat the step above to take apart $(b\cdot(! c))$, introducing yet another \emph{X} variable.
\begin{align}
	(b\cdot(! c)): X_1 \Rightarrow  \nonumber\\
	& b : \; (X_2 \rightarrow X_1) \nonumber\\
	& (! c) :\;  X_2
\end{align}

Now $b$ has been isolated as well, leaving $(! c)$ as the only unresolved proof constant. The introspection operator in this proof term results in a new \emph{X} variable, which combined with the proof constant, replaces the previous \emph{X} variable.
\begin{equation*}
	\begin{split}
		(! c) : X_2 \Rightarrow \\
		& X_2 = (c:X_3)
	\end{split}	
\end{equation*}

This gives us all the \emph{musts} for $a_2$. As can be seen below the \emph{X} variable $X_2$ has been replaced by $(c:X_3)$.
\begin{equation}\label{eq:a2_musts}
	musts \; a_2: \quad [(a, (X_1 \rightarrow (c:F))), \;(b, ((c:X_3) \rightarrow X_1)),\; (c, X_3)]
\end{equation}

Note that the same proof constant may occur more than once in the list of \emph{musts} for one \emph{atom}. 

\subsection{Using the CS-List}
We now have to look up all \emph{musts} of an atom the see if the atom is provable.
\begin{align*}
	cs = \{& a: [(H \rightarrow (c:F)), ((E \rightarrow (c:D)) \rightarrow (c:F)), (E \rightarrow (c:D))],\\
	& b: [((c:F) \rightarrow H), ((c:D) \rightarrow (a:F)), ((H \rightarrow G) \rightarrow H), (Y_1 \rightarrow (Y_2 \rightarrow Y_1))],\\
	& c: [(c:F), G, D, (G \rightarrow F)]\}
	\tag{\ref{cs}}
\end{align*}

The atom $a_1$ (\ref{eq:a_1}) is not provable, since its only \emph{must} $c:F$ cannot be found in the cs-list.

The other atom $a_2$ (\ref{eq:a_2}) needs more work. First we select and compare all \emph{musts} of $a_2$ with the corresponding entries in the cs-list. We do this by searching for the entries of the proof constants, which double as keys to lists of terms in the cs-list. The value of the \emph{must} is then matched to the entries of the list of terms. This gives us one or more sets of conditions per proof constant.  

\subsubsection[look up proof constant a]{Proof Constant $a$}
Comparing $(X_1 \rightarrow (c:F))$ with all entries in cs-list for the proof constant $a$ will give us the following two sets of condition which correspond to only the variable $X_1$.
\begin{align}
	(H \rightarrow (c:F)) & \; \Rightarrow \; \{X_1: H\} \nonumber\\ 
	((E \rightarrow (c:D)) \rightarrow (c:F)) & \; \Rightarrow  \; \{X_1: (E \rightarrow (c:D))\} \label{condition:a}
\end{align}

\subsubsection[look up proof constant b]{Proof Constant $b$}
For the proof constant $b$ with \emph{must}-term $((c:X_3) \rightarrow X_1)$ we get:
\begin{align}
	((c:F) \rightarrow H) & \; \Rightarrow \; \{X_1: F,\; X_3: H\}\nonumber\\ 
	((c:D) \rightarrow (a:F)) & \; \Rightarrow \; \{X_1: (a:F),\; X_3: D\}\nonumber\\ 
	((Y_1 \rightarrow (Y_2 \rightarrow Y_1)) & \; \Rightarrow \; \{X_1: (Y_2 \rightarrow Y_1),\; Y_1: (c:X_3)\} \label{condition:b}
\end{align}
We note that for the last set of conditions we now have a \emph{Y} variables aside from the \emph{X} variables given in the \emph{must} term. For the moment both kinds of variables are treated exactly the same.

\subsubsection[look up proof constant c]{Proof Constant $c$}
Since the \emph{must} term for proof constant $c$ is simply $X_3$ we get the following sets of condition.
\begin{align}
	(c:F) & \; \Rightarrow \; \{X_3: (c:F)\} \nonumber\\ 
	G & \; \Rightarrow \; \{X_3: G\} \nonumber\\ 
	D & \; \Rightarrow \; \{X_3: D\}\label{condition:c} \nonumber\\ 
	(G \rightarrow F) & \; \Rightarrow \; \{X_3: (G \rightarrow F)\} 
\end{align}

\section{Constructing the Final Result}
Now we have several condition sets for each proof constant that have to be put together to form a solution.
\subsection{Merging Conditions}
We put together a new set that contains one set of conditions per proof constant. Then we try to simplify this new set. If we encounter a contradiction in doing so, the new set is discarded. If the new set stays without contradictions, we have found a configuration.
For example we could pick the first set from each, but this would give us a contradiction since $X_3$ can only be either $H$ or $(c:F)$ but not both.

As we can see, not every set of $a$ can be successfully merged with another set of $b$. We can only take those that have the same term for $X_3$ or where there is a \emph{Y} variable. In fact only the two bottom rows are compatible, since no entry form $b$ fits $X_1: H$ from $a$ and only $(Y_2 \rightarrow Y_1)$ can be matched with $(E \rightarrow (c:D))$.
\begin{align}
	a \cap b: \; \{X_1: (E \rightarrow (c:D)), \; X_1: (Y_2 \rightarrow Y_1), \; Y_1: (c:X_3)\}
\end{align}

As seen above there are now two conditions that apply to the variable $X_1$. Before we move on and try to merge this set of conditions with one of the sets of $c$ we will resolve the current conditions if possible.

Comparing the conditions of $X_1$ we find that $Y_2: E$ and $Y_1: (c:D)$. Since we already have a condition for $Y_1$ that condition is now compared with the new one we got from $X_1$. This will give us $X_3: D$. Thus, all our variables are now configured:
\begin{align}
	\{X_1: (E \rightarrow (c:D)), \; X_3: D, \; Y_1: (c:D), \; Y_2: E\}
\end{align}

As a consequence of merging the set of conditions (\ref{condition:a}) from $a$ with set of conditions (\ref{condition:b}) from $b$ there is no choice left for any of the \emph{X} variables. The final result depends on finding a set of conditions of proof constant $c$ that doesn't contradict the value for $X_3$. As it happens this is the case for the set of conditions (\ref{condition:c}).


\subsection{Meaning of the Result}
Since we found a valid configuration for the atom $a_1$ (\ref{eq:a_1}) we have shown that the formula $f$ (\ref{eq:f}) is provable. Lets take a step back and see what the configurations of the \emph{X} variables have to do with the provability of $f$.

From our previous step we have a configuration for every variable. We are however only interested in the \emph{X} variables and do not care about the \emph{Y} variables. We know that $X_1 = (E \rightarrow (c:D))$ and $X_3 = D$. We replace the \emph{X} variables in the \emph{musts} with those values. This gives us:
\begin{equation}
\begin{split}
	a_2: \; [&(a: ((E \rightarrow (c:D)) \rightarrow (c:F))), \\
	&(b: ((c:D) \rightarrow (E \rightarrow (c:D)))), \\
	&(c: D)]
\end{split}
\end{equation}

As we see these entries can all be found exactly like that in the cs-list (\ref{cs}). From those we can also reconstruct the term of $a_2$: 
\begin{align}
	\Rightarrow & \; (c:D) \\
	\Rightarrow & \; ((!c):(c:D)) \\
	\Rightarrow & \; ((b\cdot(!c)):(E \rightarrow (c:D)))\\
	\Rightarrow & \; ((a\cdot(b\cdot(!c))):(c:F)) \label{ex:reconstruct}
\end{align}

And with (\ref{ex:reconstruct}) for $a_2$ we have again what we started with right after the atomization step in (\ref{eq:sp_iii}). In the graph below the path with the tree of the atom $a_2$ is highlighted. 
%!TEX root = ../../bachelors_thesis.tex
\begin{figure}[H]
\caption{Syntax tree of formula $f$ with atom $a_2$ hightlighted.}
\begin{center}
\begin{tikzpicture}[level distance=1.2cm,
  level 1/.style={sibling distance=5cm},
  level 2/.style={sibling distance=3.5cm}, 
  level 3/.style={sibling distance=2cm},
  level 4/.style={sibling distance=1cm},
  level 5/.style={sibling distance=1.5cm}, thick,scale=0.9, every node/.style={scale=0.9}]
	\node [arn_n]{$:$}
	  child {node [arn_r]{\(+ \)}
	    child {node [arn_w] {$!$}
	    	child[right] {node [arn_w] {$+$}
	    		child {node [arn_w] {$a$}}
	    		child {node [arn_w] {$c$}}
	    		}
	    	}
	    child {node [arn_r] {$*$}
	    	child {node [arn_r] {$+$}
	    		child {node [arn_r] {$a$}}
	    		child {node [arn_w] {$!$}
	    			child[right] {node [arn_w] {$a$}}
	    			}
	    		}
	    	child {node [arn_r] {$*$}
	    		child {node [arn_r] {$b$}}
	    		child {node [arn_r] {$!$}
	    			child[right] {node [arn_r] {$c$}}
	    			}
	    		}
	    	}
	    }
	  child {node [arn_r]{$:$}
	  	child {node [arn_r] {$c$}}
	  	child {node [arn_r] {$F$}}
	  };
\end{tikzpicture}
\end{center}
\end{figure}


This concludes the Example chapter. 