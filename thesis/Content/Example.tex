%!TEX root = ../bachelors_thesis.tex
In this chapter I walk through a complete example covering as many special cases as possible. As such, the justification term we will look at is rather complicated but it will also show how nicely it can be broken down in more simpler atoms.

\section{Initialization}
The algorithm takes a formula $f$ and a cs-list as input. The data presented here is in the same form as it would be entered into the program. Therefore the cs-list is a \emph{python} dictionary and not simple list of pairs and there are more brackets explicitly written then required by convention.

\begin{align}\label{eq:f}
f = (((!(a+c))+((a+(!a))\cdot (b\cdot (!c)))):(c:F))
\end{align}

\begin{equation}\label{cs}
\begin{split}
	cs = \{& a: [(H \rightarrow (c:F)), ((E \rightarrow (c:D)) \rightarrow (c:F)), (E \rightarrow (c:D))],\\
	& b: [((c:F) \rightarrow H), ((c:D) \rightarrow (a:F)), ((H \rightarrow G) \rightarrow H), (Y_1 \rightarrow (Y_2 \rightarrow Y_1))],\\
	& c: [(c:F), G, D, (G \rightarrow F)]\}
\end{split}
\end{equation}




\section{Walking in Trees: Atomize}

The given formula $f$ is transformed into a syntax tree using \texttt{parse\_formula} of \texttt{Tree}. 

\subsection{Sumsplit}
Our first step is to split our formula for every sum we encounter.
\begin{equation*}
	(((!(a
    \tikz[baseline]{\node[fill=red!20,anchor=base] (t1){$+$};} c))
    \tikz[baseline]{\node[fill=red!20,anchor=base] (t2){$+$};} ((a
    \tikz[baseline]{\node[fill=red!20,anchor=base] (t3){$+$};} (!a))\cdot (b\cdot (!c)))):(c:F))
\end{equation*}

%!TEX root = ../../bachelors_thesis.tex
\begin{figure}[H]
\begin{center}
\begin{tikzpicture}[level distance=1.2cm,
  level 1/.style={sibling distance=5cm},
  level 2/.style={sibling distance=3.5cm}, 
  level 3/.style={sibling distance=2cm},
  level 4/.style={sibling distance=1cm},
  level 5/.style={sibling distance=1.5cm}, thick,scale=0.9, every node/.style={scale=0.9}]
	\node [arn_n]{$:$}
	  child {node [arn_r]{\(+ \)}
	    child {node [arn_w] {$!$}
	    	child[right] {node [arn_r] {$+$}
	    		child {node [arn_w] {$a$}}
	    		child {node [arn_w] {$c$}}
	    		}
	    	}
	    child {node [arn_w] {$*$}
	    	child {node [arn_r] {$+$}
	    		child {node [arn_w] {$a$}}
	    		child {node [arn_w] {$!$}
	    			child[right] {node [arn_w] {$a$}}
	    			}
	    		}
	    	child {node [arn_w] {$*$}
	    		child {node [arn_w] {$b$}}
	    		child {node [arn_w] {$!$}
	    			child[right] {node [arn_w] {$c$}}
	    			}
	    		}
	    	}
	    }
	  child {node [arn_w]{$:$}
	  	child {node [arn_w] {$c$}}
	  	child {node [arn_w] {$F$}}
	  };
\end{tikzpicture}
\caption{Syntax tree of given formula $f$ before it is atomized.}
\end{center}
\end{figure}


The \texttt{sum\_split} from \texttt{Tree} will give us the following terms in form of a list. 

\begin{equation}\label{eq:sp_i}	
	((!a):(c:F))
\end{equation}
\begin{equation}\label{eq:sp_ii}	
	((!c):(c:F))
\end{equation}	
\begin{equation}\label{eq:sp_iii}	
	((a \cdot(b\cdot (! c))):(c:F))											
\end{equation}
\begin{equation}\label{eq:sp_iv}	
	(((! a)\cdot(b\cdot (! c))):(c:F))												
\end{equation}
Some of those will eventually be the atoms for $f$, but before they can become atoms of $f$ they need to be simplified and checked.

\subsection{Introspections}
We now look at the terms to discard any that have a introspection operation that is the left child of a multiplication or if there are proof terms that start with a introspection operation and cannot be simplified. If they can be simplified they will of course not be discarded but simplified.

\subsubsection[First term]{Term (\ref{eq:sp_i})}
In this term we find a introspection which is valid, since it is not a left child of a multiplication, but trying to simplify the term shows us that it cannot be resolved thus letting us discard this term.

\begin{equation*}
	((\tikz[baseline]{\node[fill=red!20,anchor=base] (t1){$!$};}a):(c:F)) 
\end{equation*}

\subsubsection[Second term]{Term (\ref{eq:sp_ii})}
As before the introspection for the term is valid and in contrast to the previous example the term here can be simplified, giving us our first \emph{atom} for formula $f.$

\begin{equation}\label{eq:a_1}
	\begin{split}
	((\tikz[baseline]{\node[fill=red!20,anchor=base] (t1){$!$};} c):(c:F)) & \Rightarrow \\
	& a_1 := (c:F)
	\end{split}
\end{equation}

\subsubsection[Third term]{Term (\ref{eq:sp_iii})}
In this term we find the introspection operation neither a left child of a multiplication nor as top operation of the proof term and thus we have our second \emph{atom}.

\begin{equation}\label{eq:a_2}
	\begin{split}
	((a \cdot(b\cdot (\tikz[baseline]{\node[fill=red!20,anchor=base] (t1){$!$};} c))):(c:F))	 & \Rightarrow \\
	& a_2 := ((a \cdot(b\cdot (! c))):(c:F))
	\end{split}
\end{equation}

\subsubsection[Fourths term]{Term (\ref{eq:sp_iv})}
Finally this term has two introspections of which the first is the left child of a multiplication and thus makes the term invalid. The second introspection would be valid, but the first term causes the whole subterm to be discarded.

\begin{equation*}
	(((
	\tikz[baseline]{\node[fill=red!20,anchor=base] (t1){$!$};} a)\cdot(b\cdot (
	\tikz[baseline]{\node[fill=red!20,anchor=base] (t1){$!$};} c))):(c:F))
\end{equation*}

\bigskip
This completes the \texttt{atomize} step for the formula $f$ giving us the two atoms $a_1$ and $a_2$. Showing that at least one of those is provable is enough to show that $f$ is provable. 

\section{Getting and Looking up the Musts}

\begin{equation*}
		f = (((!(a+c))+((a+(!a))\cdot(b\cdot(!c)))):(c:F)) 
		\tag{\ref{eq:f}}  
\end{equation*}
We have found the two atoms $a_1$and $a_2$ for the formula $f$. The next step determins the \emph{musts} if needed, matching them against the cs-list and finally merge the possible configurations together to determine if one of the musts is provable.


\begin{equation*}
		a_1 = (c:F) 
		\tag{\ref{eq:a_1}}
\end{equation*}
\begin{equation*}		
		a_2 = ((a \cdot(b\cdot (! c))):(c:F)) 
		\tag{\ref{eq:a_2}}
\end{equation*}

\subsection{Musts}
\subsubsection[First atom]{Atom $a_1$ (\ref{eq:a_1})}
Since $a_1$ consists already only of one proof constant with the corresponding term there is nothing further to to here.
\begin{equation}
	a_1: \quad musts = [(c, F)]
\end{equation}

\subsubsection[Second atom]{Atom $a_2$ (\ref{eq:a_2})}
For $a_2$ we need to take the proof term apart bit by bit. The first operation we take apart is a application. Extracting proof constants from a application proof term us give a \emph{X-wild}. 

Whenever a new X variable appears the $i$ of $X_i$ will simply be increased by one to ensure that it is fresh in the formula.

\begin{align*}\label{eq:musts1_a_2}
		((a \cdot(b\cdot (! c))):(c:F)) \Rightarrow & \\
		 a : &(X_1 \rightarrow (c:F)) \\
		 (b\cdot(! c)): &X_1
\end{align*}

The proof constant $a$ has been isolated but $(b\cdot(! c))$ still needs to be taken apart further. We repeat the step from above and introduce yet another X variable.

\begin{align*}
	(b\cdot(! c)): X_1 \Rightarrow & \\
	b : & (X_2 \rightarrow X_1) \\
	(! c) : & X_2
\end{align*}

Now $b$ has been isolated as well, leaving $(! c)$ as the only unresolved proof constant. Having a introspection in this situation results in a new X variables in combination with the proof term which will replace a previous X variables.

\begin{equation*}
	\begin{split}
		(! c) : X_2 & \Rightarrow \\
		& X_2 = (c:X_3)
	\end{split}	
\end{equation*}

This finally gives us all the \emph{musts} for $a_2$. As can be seen belove the \emph{X-wild} $X_2$ has been replaced by $(c:X_3)$.
\begin{equation}\label{eq:a2_musts}
	[(a, (X_1 \rightarrow (c:F))), (b, ((c:X_3) \rightarrow X_1)), (c, X_3)]
\end{equation}

Note that the same proof constant may be in more than one of the \emph{musts} for one \emph{atom}. 

\subsection{Using the CS-List}
We now have to look up all \emph{musts} of each atom the see if the atom is provable.

\begin{align*}
	cs = \{& a: [(H \rightarrow (c:F)), ((E \rightarrow (c:D)) \rightarrow (c:F)), (E \rightarrow (c:D))],\\
	& b: [((c:F) \rightarrow H), ((c:D) \rightarrow (a:F)), ((H \rightarrow G) \rightarrow H), (Y_1 \rightarrow (Y_2 \rightarrow Y_1))],\\
	& c: [(c:F), G, D, (G \rightarrow F)]\}
	\tag{\ref{cs}}
\end{align*}

The atom $a_1$ (\ref{eq:a_1}) is not provable, since its only \emph{must} $c:F$ cannot be found in the cs-list.

The other atom $a_2$ (\ref{eq:a_2}) needs a little bit more work. First we select and compare all \emph{musts} of $a_2$ with the corresponding entries in the cs-list and then we need to find a configuration for the variables of the musts, that will fit all \emph{musts}.

\subsubsection[look up proof constant a]{Proof Constant $a$}
Comparing $(X_1 \rightarrow (c:F))$ with all entries in cs-list for the proof constant $a$ will give us the following two condition sets which are only on the variable $X_1$.
\begin{align}
	(H \rightarrow (c:F)) & \quad \Rightarrow \quad \{X_1: H\} \\ 
	((E \rightarrow (c:D)) \rightarrow (c:F)) & \quad \Rightarrow  \quad \{X_1: (E \rightarrow (c:D))\} \label{condition:a}
\end{align}

\subsubsection[look up proof constant b]{Proof Constant $b$}
For the proof constant $b$ with \emph{must}-term $((c:X_3) \rightarrow X_1)$ we get:
\begin{align}
	((c:F) \rightarrow H) & \quad \Rightarrow \quad \{X_1: F, & X_3: H\}\\ 
	((c:D) \rightarrow (a:F)) & \quad \Rightarrow \quad \{X_1: (a:F), & X_3: D\}\\ 
	((Y_1 \rightarrow (Y_2 \rightarrow Y_1)) & \quad \Rightarrow \quad \{X_1: (Y_2 \rightarrow Y_1), & Y_1: (c:X_3)\} \label{condition:b}
\end{align}
We note that for the last condition set we now have second kind of variable aside from those given in the \emph{must} term. For the moment both kinds of variables are treated exactly the same.

\subsubsection[look up proof constant c]{Proof Constant $c$}
Since the must term for proof constant $c$ is simply $X_3$ we get the following condition sets.
\begin{align}
	(c:F) & \quad \Rightarrow \quad \{X_3: (c:F)\} \\ 
	G & \quad \Rightarrow \quad \{X_3: G\} \\ 
	D & \quad \Rightarrow \quad \{X_3: D\}\label{condition:c} \\ 
	(G \rightarrow F) & \quad \Rightarrow \quad \{X_3: (G \rightarrow F)\} 
\end{align}

\section{Constructing the Final Result}
Now we have several condition sets for each proof constant that have to be put together to a solution.
\subsection{Merging Conditions}
Our goal is to pick one line from each proof constant and that this merged conditions give us a configuration for the X variables. For example we could pick from each the top line, but it is obvious that this is not a solution since $X_3$ can only be either $H$ or $(c:F)$ but not both.

It is clear that not every line of $a$ can be successfully merged with every line of $b$. We see that we can only take those that have the same term for $X_3$ or there is a $Y$-variable. If fact only the two bottom row are compatible, since no entry form $b$ fits $X_1: H$ from $a$ and only $(Y_2 \rightarrow Y_1)$ can be matched with $(E \rightarrow (c:D))$.

\begin{align}
	a \cap b: \quad \{X_1: (E \rightarrow (c:D)), \quad X_1: (Y_2 \rightarrow Y_1), \quad Y_1: (c:X_3)\}
\end{align}

As seen above there are now two conditions that apply to the variable $X_1$. Before we move on and try to merge this set of conditions with one of the lines of $c$ we will resolve the current conditions as far as possible.

Comparing the conditions for $X_1$ we find that $Y_2: E$ and $Y_1: (c:D)$. Since we have already a condition for $Y_1$ that condition is now compared with the new we got from $X_1$ and we will get $X_3: D$. Thus all our variables are now configured:

\begin{align}
	\{X_1: (E \rightarrow (c:D)), \quad X_3: D, \quad Y_1: (c:D), \quad Y_2: E\}
\end{align}

As a consequence of merging line (\ref{condition:a}) from $a$ with line (\ref{condition:b}) from $b$ there is no choice left for the variable and the final result depends on finding a line from proof constant $c$ that matches the value for $X_3$ and as it happens this is the case for line (\ref{condition:c}).


\subsection{Meaning of the Result}
Since we found a valid configuration for the atom $a_1$ (\ref{eq:a_1}) we have shown that the formula $f$ (\ref{eq:f}) is provable. Lets take a step back and see, what the X variables have to do with the provability of $f$.

From our previous step we have a configuration for every variable. We are however only interested in the X variables and do not care further about the $Y$-variables. So we know that $X_1 = (E \rightarrow (c:D))$ and $X_3 = D$. If we replace that in the \emph{musts} for all of the proof constants we get the following:

\begin{equation}
\begin{split}
	a_2: \quad [&(a: ((E \rightarrow (c:D)) \rightarrow (c:F))), \\
	&(b: ((c:D) \rightarrow (E \rightarrow (c:D)))), \\
	&(c: D)]
\end{split}
\end{equation}

As can be seen these entries can all be found precisely like that in the cs-list. Also from those we can reconstruct the term of $a_2$: 

\begin{align}
	& (c:D) \\
	& ((!c):(c:D)) \\
	& ((b\cdot(!c)):(E \rightarrow (c:D)))\\
	& ((a\cdot(b\cdot(!c))):(c:F)) \label{ex:reconstruct}
\end{align}

And with (\ref{ex:reconstruct})) for $a_2$ we have again with what we started right after the atomization step in (\ref{eq:sp_iii}). In the graph below the path with the tree of the atom $a_2$ is highlighted.

%!TEX root = ../../bachelors_thesis.tex
\begin{figure}[H]
\caption{Syntax tree of formula $f$ with atom $a_2$ hightlighted.}
\begin{center}
\begin{tikzpicture}[level distance=1.2cm,
  level 1/.style={sibling distance=5cm},
  level 2/.style={sibling distance=3.5cm}, 
  level 3/.style={sibling distance=2cm},
  level 4/.style={sibling distance=1cm},
  level 5/.style={sibling distance=1.5cm}, thick,scale=0.9, every node/.style={scale=0.9}]
	\node [arn_n]{$:$}
	  child {node [arn_r]{\(+ \)}
	    child {node [arn_w] {$!$}
	    	child[right] {node [arn_w] {$+$}
	    		child {node [arn_w] {$a$}}
	    		child {node [arn_w] {$c$}}
	    		}
	    	}
	    child {node [arn_r] {$*$}
	    	child {node [arn_r] {$+$}
	    		child {node [arn_r] {$a$}}
	    		child {node [arn_w] {$!$}
	    			child[right] {node [arn_w] {$a$}}
	    			}
	    		}
	    	child {node [arn_r] {$*$}
	    		child {node [arn_r] {$b$}}
	    		child {node [arn_r] {$!$}
	    			child[right] {node [arn_r] {$c$}}
	    			}
	    		}
	    	}
	    }
	  child {node [arn_r]{$:$}
	  	child {node [arn_r] {$c$}}
	  	child {node [arn_r] {$F$}}
	  };
\end{tikzpicture}
\end{center}
\end{figure}


\bigskip
This concludes this chapter showing as much as possible with a concise example.